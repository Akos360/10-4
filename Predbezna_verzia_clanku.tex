\documentclass[10pt,twoside,english,a4paper]{article}

\usepackage[slovak]{babel}
\usepackage{fontenc} 
\usepackage{inputenc}
\usepackage{graphicx}
\usepackage{url} 
\usepackage{hyperref} 
\usepackage{cite}

\pagestyle{headings}
\centering
\title{Závislosť študentov na hrách
\thanks{Semestrálny projekt v predmete Metódy inžinierskej práce, ak. rok 2022/23, vedenie: Ákos Lévárdy}}

\author{Ákos Lévárdy\\[2pt]
	{\small Slovenská technická univerzita v Bratislave}\\
	{\small Fakulta informatiky a informačných technológií}\\
	{\small \texttt{xlevardya@stuba.sk}}	
	}

\date{\small 5. november 2022}


\begin{document}

\maketitle

\begin{abstract}

Videohra môže byť akýkoľvek softvérový program, ktorý je možné hrať na výpočtovom zariadení, ako je osobný počítač, mobilný telefón alebo herná konzola. Videohry sú veľmi populárne a priateľské k veku a pohlaviu, ale na druhej strane môžu mať nepriaznivé následky, ktoré sa skúmajú. V mojom článku sa chcem venovať na účinky hrania na zdravie, porovnáva hranie s inými závislosťami a otvára ďalšiu diskusiu o poruchách hrania. Celkovým cieľom je preskúmať metódy prevencie herných porúch vrátane vzdelávania, úlohy rodiny a životného štýlu. Hranie videohier má mnoho charakteristických znakov iných porúch závislosti vrátane škodlivých následkov pre fyzické a duševné zdravie. Sociálna a všadeprítomná povaha hrania hier sťažuje rozpoznanie príznakov a symptómov nadmerného hrania hier. Výskum závislosti naďalej preukazuje podobnosti medzi príznakmi návykového herného správania a závislosťami súvisiacimi s látkami.

\end{abstract}

\pagebreak

\section*{Úvod}

Masívne multiplayerové online hry na hranie rolí MMORPG ~\ref{def}, sú jednou z najrýchlejšie rastúcich foriem závislosti na internete, najmä medzi deťmi a dospievajúcimi. Podobne ako závislosť od alkoholu alebo drog, aj hráči vykazujú niekoľko klasických príznakov závislosti ~\ref{zavis}.\\
Nadmerné hranie môže viesť ku zdravotným problémom ~\ref{zdrav_prob}, ale ak existuje problém, tak existuje k nemu aj riešenie~\ref{ries}.\\
Záverečné poznámky~\ref{zaver}.

\section{Definícia} \label{def}
 
Online videohry sú veľmi populárnou prevládajúcou formou zábavy. Denne viac ako 1 000 000 ľudí hrá rôzne hry na internete. V dnešnej dobe hrať videohru nezávisí od veku ani od pohlaviu, práve preto počet ľudí sa iba zvišuje.
Sú digitálne hry a online prostredie návykové? Problém spočíva v tom, že prostredie internetu a digitálne hry ponúkajú ľudskému mozgu neustálu odmenu. Ak sa o niečo snažíme alebo sa niekam chceme dostať v skutočnosti, musíme mať dobrú motiváciu a tiež musíme vynaložiť určitú energiu na to, aby sme dosiahli odmenu. Napríklad ak si niekto ide zabehať, stojí ho to veľa námahy. Keď sa ale dostane k svôjmu cieľu, jeho mozog uvoľní látky, ktoré ho týmto spôsobom odmenia, že to splnil alebo dokázal. V online prostredí sa k odmene dostávame skratkou.
Hry a aplikácie v človeku vyvolávajú chemické reakcie. Človek neustále myslí na hranie, ktoré sa postupne stáva centrom jeho života. Naše nálady sa menia a hra to celkom oplyvňuje. Napríklad keď niekto je smutný tak si zahrá niečo aby sa cítiľ lepšie, ale na opačnej strane keby niekto bol šťastný, tak si zahrá kvôli tomu, že si to chce osláviť. Veľmi veľa ludí robí práve to isté, len nie cez hry ale cez sociálne siete ako Youtube, Instagram alebo TikTok.

\section{Závislosť} \label{zavis}

Návykovosť je spôsobená tým, ako sa hry aj napríklad sociálne siete vytvárajú. 
Samotné hry nemajú dostatočnú silu vyvolať závislosť, pokiaľ daný hráč nemá určité vlastnosti. 
Závislosť znamená hrať hry každodenne viac hodín a väčšinou nie kvôli svojej vôle, ale kvôli pocitu, že potrebujem hrať a nebyť schopný dobrovoľne prestať. Závislosť ovláda správanie sa človeka. Otázkou je prečo práve hráme hry? Najprv mnoho ľudí začne hrať totiž je to dočasný únik od reality. Tým pádom na určitý čas nemusia sa starať so svojimi povinnosťami. Na druhej strane väčšina hier je skôr sociálna než individuálna. Na online platforme každý ukazuje len svoj obraz, len to čo chce aby ostatný hráči videli. Netreba každému povedať svoje problémy. Spoluhráči nerozmýšlajú o tom, že ako práve vyzeráme, odkiaľ sme, čo pracujeme či aký jazyk rozprávame.
Každá hra nám dá určitú výzvu, čo musíme zvládnuť aby sme sa dostali do cieľu. Na konci vždy nájdeme odmenu, totiž bez úspechu by sme ani nezačali hrať tú hru. Mozog človeka funguje tak, že vždy chce niečo zvládnuť, niekam sa dostať. Nikdy neprestaneme učiť sa, iba treba nájsť riadnu motiváciu. Hráč ktorý nevie dobrovoľne prestať hrať si skôr predstavuje že on žije v online svete. Pre takého hráča realita je veľmi krutá a je oveľa ľachšie ovládať svôj charakter cez hru, cez internet než v reálnom živote.

\section{Zdravotné problémy} \label{zdrav_prob}

Ako som už spomenul každodenná hra sa používa ako únik zo sveta alebo relax. Hráč postupne je viac a viac pod tlakom, že potrebuje hrať. Viac hrania preňho ale neznamená aj viac zábavy, skôr naopak. Hrá sa aj napriek tomu, že ho to nudí. Nehranie je však ešte horšie. Závislý hráč, klame o svojom hraní, stráca záujem o iné činnosti len preto, aby mohol hrať, odťahuje sa od rodiny a priateľov, aby mohol hrať, a používa hry ako prostriedok psychického úniku. Môže mať príznaky ako napríklad zlý zrak, totiž celý čas pozerá na obrazovku. Nechodieva von a nie je zviknutý na vonkajšie účinky. Ľahko ochorie kvôli tomu, že jeho imunný systém nie je v kontakte s rôznými vírusmi, chorobami. Jeho emócie sa rýchlo menia, pokiaľ takýto človek nemôže hrať, je podráždený, nahnevaný alebo nepokojný. Jeho tolerancia sa časom zvyšuje, čo znamená, že potrebuje viac času alebo intenzívnejšie hranie, aby sa dostavil pocit satisfakcie. Ak je dlho nútený nehrať alebo nebyť v prostredí internetu dostáva abstinenčné príznaky. V tomto prípade hovoríme o nervozite, smútku, podráždenosti, agresivite, úzkosti či depresie. 
Neskôr vznikajú konflikty s priateľmi, partnerom, rodičmi či kolegami. Nemusí byť taký konflikt s inými ľudmi ale môže mať aj so sebou samým. Je to typické pre študentov, ktorý na jednej strane musia napísať seminárku, na druhej strane ich láka hranie, ktoré väčšinou nakoniec vyhráva. Vnútorný konflikt je nepríjemný a často sa rieši ďalším hraním. Relaps nastáva vtedy, keď sa hráč pokúsi o zníženie činnosti, lebo vie, že mu už prináša negatíva, ale po istom čase sa k nej znovu vráti. 
Vplyv problematického používania internetu na duševné zdravie, najmä na depresiu mladých ľudí je dôležitá téma. Vidno možné cesty prepojenia medzi závislosťou od internetových hier a depresiou pravdepodobne sprostredkovanou problémami so spánkom. Výsledky prehľadu naznačujú, že návykové hry, môžu byť spojené s horšou kvalitou spánku. Problematické používanie internetu je spojené s problémami so spánkom vrátane subjektívnej nespavosti a zlej kvality spánku.~\cite{Internet_Gaming_Addiction}

\section{Riešenia na závislosť} \label{ries}

Existuje viac riešení na to ako znížiť závislosť študentov na hrách. Prvou najrozumnejšou možnosťou je nahradiť niaky čas čo by sme trávili s hraním s inou aktivitou ktorá je viac produktívna. Produktívnou činnosťou môže byť napríklad učenie sa, napísanie úlohy, upratovanie alebo aj venčenie psa. Dôležité je si nahrať alebo merať koľko času trávime na danú hru. Potrebujeme realizovať práve koľko iných vecí by sme mohli dosiahnuť namiesto hrania. Musíme vidiet spätnú vezbu. Ďaľšou možnosťou je odstrániť applikáciu, na tvrdo vymazať hru aby “nestrašila“. Zbaviť sa konzoly, lebo keď ju nevidíme, tak ani necítime k nej vzťah. Tiež je možnosť vytvoriť si pravidlo “ak-potom”, čo znamená ak vykonám produktívnu vec, potom si môžem zahrať hru pre daní čas. Limitovanie sa vedie k lepšej rovnováhe, vieme čo máme urobiť pre to, aby sme dosiahli cieľ. ~\cite{Gaming_Addiction_Treatment}

\section{Záver} \label{zaver}

Tento článok sa zaoberá so vznikom závislosti od online hier a jej vplyv na hráčov a rodiny. V tomto dokumente sa skúma povaha online hier a to, čo spôsobuje, že niektorí hráči sú závislí. Tiež sa týka viacerých riešení na závislosti hrania videohier. Otvára ďalšiu otázku ako nadmerné hranie hier oplyvňuje vzdelávanie.

\bibliography{literatura}
\bibliographystyle{abbrv}

\end{document}
